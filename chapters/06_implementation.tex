\chapter{Implementation}
\label{ch:implementation}

Our implementation of \projectname{} is integrated into the LLVM framework, wasi-libc, and the wasmtime WebAssembly runtime.
The following subsections detail the specific modifications and extensions we made to each component.

\section{LLVM Extensions}
\label{sec:llvm}

In LLVM, we introduced a novel sanitizer pass, that can be enabled via a compiler flag, designed to provide memory safety for stack allocations when compiling to WebAssembly.
This sanitizer analyzes functions for stack allocations and applies padding and tagging to them.
This ensures spatial memory safety by ensuring neighbouring allocations are tagged with different tags and temporal safety by untagging the stack frame when returning from a function.
As WebAssembly does not support exceptions or C-style long jumps, we do not have to handle these special cases, which have proved to be tricky to get right with tagged memory.

We extended LLVM with new built-in functions, which can be invoked directly from C code.
These functions provide programmers direct access to our newly introduced memory tagging primitives, allowing for fine-grained control over memory operations.
Utilizing these built-in functions, we modified the default allocator in wasi-libc to provide memory safety for heap allocations.

\section{WASI Libc Modifications}
\label{sec:wasi-libc}

To allow us to run applications relying on libc on wasm64, we had to port the WebAssembly System Interface (WASI) and wasi-libc to wasm64.
This mainly consisted of mechanical work, changing size and pointer types to 64\,bits.

To provide memory safety for heap allocations, we modified dlmalloc, the default allocator in wasi-libc.
Before returning allocated memory to the user, we tag the memory.
This has the additional effect of not only protecting adjacent allocations from being accessed via memory overflows, but also the allocator metadata itself.
When memory is freed or reallocated, we untag the memory regions returned to the allocator.
This provides temporal safety, even after freed memory is allocated again, as the tag for the new allocation will be chosen at random.

\section{WebAssembly Runtime Enhancements}
\label{sec:wasm-runtime}

Our implementation in the wasmtime runtime focuses on combining performance with security.
In environments where hardware supports MTE, our new instructions are lowered to MTE instructions, leveraging the hardware's capabilities for memory safety.
We have implemented a number of optimizations to ensure our generated code runs efficiently.
In the general case, we generate a loop over the memory region to be tagged.
This loop, however, is unrolled to tag larger blocks of memory at once, and is unrolled to a fixed number of instructions, if the size of the memory region to be tagged is known at compile time.
