\chapter{Implementation}
\label{ch:implementation}

Our implementation of our prototype is integrated into the LLVM framework, wasi-libc, and the wasmtime WebAssembly runtime.
The following sections detail the specific modifications and extensions we made to each component, as well as some of the implementation choices and details to tackle specific problems.

\section{LLVM}
\label{sec:llvm}

We chose LLVM as our compiler from C/C++ to WebAssembly.
To be able to compile to our memory safety extension, we modified the WebAssembly backend to add support for our new instructions, allowing LLVM to emit our extension in both bytecode and text format.

\subsection{LLVM IR}
\label{subsec:llvm-ir}

In the middle end, we introduced three new intrinsic functions that correspond and are lowered to our \ac{WASM} instructions by the backend.
\begin{itemize}
  \item \lstinline[style=customc,language=llvm]{ptr @llvm.wasm.segment.new(ptr, i64);}
  \item \lstinline[style=customc,language=llvm]{void @llvm.wasm.segment.set_tag(ptr, ptr, i64);}
  \item \lstinline[style=customc,language=llvm]{void @llvm.wasm.segment.free(ptr, i64);}
\end{itemize}
Calls to these intrinsic functions can be inserted by the frontend or a sanitizer pass.
A function that allocates 32 bytes on the stack might be lowered to the following IR:

\begin{lstlisting}[frame=h,style=customc,
    label={lst:llvm-intrinsics},language=llvm]
define hidden signext void @foo(i32 %index) {
entry:
  %arr = alloca [32 x i8], align 16
  ; create a new segment
  %1 = call ptr @llvm.wasm.segment.new(ptr %arr, i64 32)

  ; do some work

  ; return ownership of segment to tag
  call void @llvm.wasm.segment.set.tag(ptr %1, ptr %arr, i64 32)
  ret void
}
\end{lstlisting}

\subsection{LLVM Sanitizer Pass}
\label{subsec:llvm-sanitizer-pass}

In LLVM, we introduced a WASM-specific sanitizer pass that can be enabled via a compiler flag, designed to provide memory safety for stack allocations when compiling to WebAssembly.
This sanitizer analyzes functions for stack allocations and applies padding and tagging to them, as discussed in \cref{subsec:stack-safety}.
The pass runs after all optimizations, ensuring we do not block passes that might remove stack allocations, such as \texttt{mem2reg}.

As WebAssembly does not support exceptions or C-style long jumps, we do not have to handle these special cases.

\subsection{C extension}
\label{subsec:c-extension}

To be able to create and manipulate segments manually, e.g.\ to build a segment-aware memory allocator, we need to expose some primitives to C.
We do this by creating three builtin functions in clang that are lowered to intrinsic calls in LLVM.

\begin{itemize}
  \item \lstinline[style=customc]{void *__builtin_wasm_segment_new(void *, unsigned long);}
  \item \lstinline[style=customc]{void __builtin_wasm_segment_set_tag(void *, void *, unsigned long);}
  \item \lstinline[style=customc]{void __builtin_wasm_segment_free(void *, unsigned long);}
\end{itemize}

The functions can be used as regular functions in C code, e.g.:

\begin{lstlisting}[frame=h,style=customc,
  label={lst:builtin-functions}]
void *my_malloc(unsigned long size) {
  void *memory = malloc(size);
  return __builtin_wasm_segment_new(memory, size);
}
\end{lstlisting}

\section{WASI Libc Modifications}
\label{sec:wasi-libc}

To allow us to run applications relying on libc on wasm64, we ported the \acf{WASI} and wasi-libc to wasm64.
This mainly consisted of mechanical work, changing size and pointer types from 32 to 64\,bits.

To provide memory safety for heap allocations, we modified dlmalloc, the default allocator in wasi-libc.
This consisted of inserting calls to our builtin functions as necessary, creating memory segments and returning tagged pointers instead of returning the pointer to the just allocated piece of memory.
This protects both allocator metadata and adjacent allocations from being accessed or modified through heap overflows.
The allocator also ensures that segments are freed when freeing or reallocating memory, ensuring temporal safety.

\section{Wasmtime}
\label{sec:wasm-runtime}

As our runtime to build our prototype, we chose wasmtime\footnote{\url{https://wasmtime.dev/}}, a WebAssembly runtime with a focus speed and correctness, written in Rust, with its own optimizing compiler, cranelift\footnote{\url{https://cranelift.dev/}}, which also features its own \ac{IR}, \ac{CLIF}.

We modified wasmtime, as well as its supporting libraries, and extended it with support to parse and process the memory safety extension described in \cref{sec:wasm-extension}.
We added support for \ac{MTE} in the form of new instructions and lowering rules to cranelift, allowing wasmtime to generate \ac{MTE} instructions when compiling for a target that supports them.

By default, the memory is tagged with one of the tags described in \cref{tab:default-tag}.

\begin{table}
  \centering
  \begin{tabular}{c | c || c}
    \textbf{Memory Safety} & \textbf{MTE Sandboxing} & \textbf{Default Tag} \\
    \hline
    No & No & $t \in \{\,0\,\}$ \\
    Yes & No & $t \in \{\,0\,\}$ \\
    Yes & Yes & $t \in \{\,1\,\}$ \\
    No & Yes & $t \in \{\,1, \dots, 15\,\}$
  \end{tabular}
  \caption{Default tag $t$ for the linear memory}
  \label{tab:default-tag}
\end{table}

For \ac{MTE}, we store the logical tag directly in the \ac{WASM} index.
When assigning the allocation tag, however, we need to translate the index to an address.
Before assigning tags, we also need to perform explicit bounds checks, otherwise we would allow untrusted guest code to set arbitrary tags in the processes address space.
In the following paragraphs we will describe how we lower each of our instructions to an \ac{MTE} backend.

\paragraph{\lstinline[style=customwasm]{segment.new}} To create a new segment, we (1) check that the requested segment is inside the linear memory for the guest, (2) generate a random logical tag and insert it into the index, and (3) set the allocation tag for the segment.
This involves generating a loop that iterates over the size of the segment and setting the tag using \texttt{stzg}, which also zeroes the memory.

\paragraph{\lstinline[style=customwasm]{segment.set_tag}} To change ownership of a segment, we again (1) check that the requested segment is inside the linear memory for the guest and (2) set the new allocation tag for the segment.
Here we do not need to create a new, random tag, as we are passed a predefined tag.

\paragraph{\lstinline[style=customwasm]{segment.free}} To invalidate a segment, we (1) check that the requested segment is inside the linear memory for the guest and (2) set the default allocation tag for the segment.
The default tag depends on the configuration and can be taken from \cref{tab:default-tag}.

\paragraph{}
We have implemented a number of optimizations to ensure our generated code runs efficiently.
When setting the allocation tag for a segment, we generate a loop iterating over the size of the segment.
If the size of the loop is known at compile time, we unroll the loop to tag a larger amount of tags per iteration to avoid branch instructions.
This is only done up to a size of 160 bytes, which we chose as a tradeoff between code size and reducing the number of branch instructions executed.

When setting an allocation tag, we have three choices:
\begin{enumerate}
  \item \texttt{stg}: Setting the tag for a single tag granule,
  \item \texttt{st2g}: Setting the tag for two tag granules,
  \item \texttt{stzg}: Setting the tag for a single granule and zeroing the granule.
  \item \texttt{stgp}: Setting the tag for a single granule and storing a pair of registers.
\end{enumerate}

To ensure new segments are always zeroed, options 1) and 2) would require an additional memset.
We benchmarked all four variants (see \cref{sec:mte-performance-evaluation}), and chose \texttt{stzg}, as it had the lowest overhead, while clobbering one less register compared to \texttt{stgp}.

\paragraph{}
If we are running the mode where both \ac{MTE} bounds checks and memory safety are enabled, we need a way to ensure only even tags are generated by instructions such as \texttt{irg} (\textbf{i}nsert \textbf{r}andom ta\textbf{g}) or \texttt{addg} (\textbf{add} ta\textbf{g}).
This can be either specified as an optional immediate to the instruction via an exclude mask, or through an include mask via a \texttt{prctl} call, which sets the corresponding register in kernel space.
We chose the latter option and set the include mask at startup time.

% TODO: write about MTE sync/async mode? We always use sync mode
