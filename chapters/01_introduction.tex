%-------------------------------------------------------------------------------
\chapter{Introduction}
\label{ch:intro}
%-------------------------------------------------------------------------------

In recent years, WebAssembly (WASM)~\cite{haas2017bringing} has gained prominence as a versatile compilation target, catering not only to web-based applications but also to a broader spectrum of use cases.
At its core, WASM is engineered to serve as an efficient compilation target for high-level, compiled languages such as C and C++.
A fundamental aspect of its design is its linear, contiguous memory model.
This model is instrumental in facilitating the straightforward compilation of these languages.
Moreover, it plays a pivotal role in ensuring the performance efficiency of the execution process.

While WebAssembly provides a sandbox for the guest code, which protects the host and other guests from malicious or buggy code, it does not inherently prevent memory safety issues within an application's memory space.
This limitation becomes particularly evident when compiling languages like C or C++, where there are no language-level guarantees to prevent these bugs.

However, recent advancements in hardware, such as ARM's Memory Tagging Extension (MTE) and the Pointer Authentication Extension (PAC), offer promising, high-performance solutions. These hardware extensions are designed to effectively address and rectify such memory safety concerns.

We introduce a comprehensive solution that leverages both MTE and PAC to address both spatial and temporal memory safety bugs in WebAssembly-compiled programs.
Our approach requires no modification to the source code.
The core contributions of this thesis are outlined as follows:

\begin{description}
    \item[WebAssembly Extension:] We develop a minimal and generic extension to the WebAssembly instruction set, allowing for protected memory regions without code changes.
    \item[Compiler Toolchain:] We develop a specialized compiler toolchain for unmodified C/C++ programs, optimized for efficiency with our WebAssembly extension.
    \item[Runtime:] We extend wasmtime to add support for our WebAssembly extension and implement it by compiling the primitives to ARM MTE.
    \item[Bounds Checks with MTE:] We leverage MTE to eliminate expensive bounds checks for 64\,bit WebAssembly programs.
    \item[Performance Evaluation:] We perform extensive evaluation of our implementation on ARM hardware, analyzing performance, memory overheads, and safety guarantees.
\end{description}
