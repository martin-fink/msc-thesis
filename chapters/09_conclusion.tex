\chapter{Conclusion}
\label{ch:conclusion}

In this thesis, we presented three pieces of work:
(1) a minimally-invasive and adaptable \ac{WASM} extension to provide memory safety primitives to compilers and programmers,
(2) an implementation consisting of (2a) a compiler toolchain integrated into LLVM, including a modified wasi-libc and allocator to provide spatial and temporal memory safety for the heap, an LLVM sanitizer pass to instrument stack allocations,
(2b) an implementation in wasmtime, compiling and running the \ac{WASM} extension on \ac{MTE} hardware, and utilizing \ac{MTE} as a replacement for software-based bounds checks,
(3) and an evaluation of our work and \ac{MTE} running on real devices.

\section{Future Work}
\label{sec:future-work}

\subsection{Additional Implementations}
\label{subsec:additional-implementations}

We have implemented a prototype based on \ac{MTE} for our memory safety extension.
However, additional implementations exploring different extensions, such as ARM's \ac{TBI}, available from ArmV8.0, would allow storing metadata in the top byte while performing access checks in software, similar to \ac{HWASAN}~\cite{serebryany2018memory}.
Software-based implementations, while slower, would allow deploying the memory safety extension to more devices than possible using recent extensions, such as \ac{MTE}.

Based on the ARM Morello board, we are working on an implementation utilizing \ac{CHERI}.
Morello provides much more fine-grained checks and unlimited compartments but requires widening pointers to 128\,bits and moving from a fixed 16\,byte alignment for segments to a dynamic alignment, depending on the size of the segment.
Additionally, as revoking capabilities is still an unsolved problem, temporal memory safety cannot be guaranteed.
Exploring and comparing these tradeoffs will be part of our future work.

\subsection{Backward Compatibility}
\label{subsec:backward-compatibility}

Currently, code compiled with the memory safety extension requires a modified runtime aware of this extension.
We are exploring techniques to embed metadata about allocations in custom \ac{WASM} segments that are ignored by runtimes but are used to provide memory safety when running on a modified runtime.

\subsection{Combining Guard Pages and \ac{MTE}}
\label{subsec:combining-guard-pages-and-mte}

Currently, we are limiting the number of sandboxes to 15, as we allocate the zero tag for the runtime and one tag per instance.
One possibility to increase the number of sandboxes would be to combine \ac{MTE} with guard pages.

\subsection{Pointer Authentication}
\label{subsec:future-work-pac}

In a previous Bachelor's thesis, \citeauthor{rehde2023wasm} worked on adding pointer authentication primitives to the \ac{WASM} extension, with an implementation using ARM's \ac{PAC}~\cite{rehde2023wasm}.
This work can be extended to be emulated in software when \ac{PAC} primitives are not available.

