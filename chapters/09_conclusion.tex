\chapter{Conclusion}
\label{ch:conclusion}

In this thesis, we presented three pieces of work:
(1) a minimally-invasive and adaptable \ac{WASM} extension to provide memory safety primitives to compilers and programmers,
(2) an implementation consisting of (2a) a compiler toolchain integrated into LLVM, including a modified wasi-libc and allocator to provide spatial and temporal memory safety for the heap, an LLVM sanitizer pass to instrument stack allocations,
(2b) an implementation in wasmtime, compiling and running the \ac{WASM} extension on \ac{MTE} hardware, and utilizing \ac{MTE} as a replacement for software-based bounds checks,
and (3) an evaluation of our work and \ac{MTE} on real hardware.


\section{Future Work}
\label{sec:future-work}

\subsection{Additional Implementations}
\label{subsec:additional-implementations}

We have implemented a prototype based on \ac{MTE} for our memory safety extension.
However, additional implementations exploring different extensions, such as ARM's \ac{TBI}, available from ArmV8.0, would allow storing metadata in the top byte while performing access checks in software, similar to \ac{HWASAN}~\cite{serebryany2018memory}.
Software-based implementations, while slower, would allow deploying the memory safety extension to more devices than possible using recent extensions, such as \ac{MTE}.

We are working on an implementation utilizing \ac{CHERI}, with an in-progress implementation for the ARM Morello development board~\cite{UCAM-CL-TR-982}.
The CHERI architecture provides much more fine-grained checks and unlimited compartments.
However, it requires widening pointers to 128\,bits and moving from a fixed 16\,byte alignment for segments to a dynamic alignment, depending on the segment size.
Additionally, revoking capabilities for temporal memory safety is more involved than \ac{MTE}~\cite{xia2019cherivoke}, where memory can be retagged.
Exploring and comparing these tradeoffs will be part of our future work.

\subsection{Backward Compatibility}
\label{subsec:backward-compatibility}

Currently, code compiled with the memory safety extension requires a modified runtime aware of this extension.
We are exploring techniques to embed metadata about allocations in custom \ac{WASM} segments that are ignored by runtimes but are used to provide memory safety when running on a modified runtime.

\subsection{Combining Guard Pages and \ac{MTE}}
\label{subsec:combining-guard-pages-and-mte}

Currently, we are limiting the number of sandboxes to 15, as we allocate the zero tag for the runtime and one tag per instance.
Future work might explore possibilities to increase the number of sandboxes by combining \ac{MTE} with guard pages.

\subsection{Pointer Authentication}
\label{subsec:future-work-pac}

In a previous Bachelor's thesis, \citeauthor{rehde2023wasm} explored the integration of pointer authentication primitives for data pointers to the \ac{WASM} extension, with an implementation using ARM's \ac{PAC} extension~\cite{rehde2023wasm}.
While \ac{WASM} lacks raw function pointers, table indices remain vulnerable to overwriting and forgery.
As we do not support intra-object safety, some overflow exploits remain possible.
These could, for instance, target an object's virtual function table to redirect control flow to a different function.

Adding support for data pointers to sign and authenticate these table indices would provide another defense against some attacks.
