\chapter{Artifacts}
\label{ch:artifacts}

The artifacts are available on GitHub.
They consist of the following repositories:

\begin{itemize}
    \item LLVM: \url{https://github.com/TUM-DSE/llvm-memsafe-wasm}
    \item wasmtime: \url{https://github.com/TUM-DSE/wasmtime-mte}
    \item wasm-tools: \url{https://github.com/TUM-DSE/wasm-tools-mte}
    \item wasi-sdk: \url{https://github.com/martin-fink/wasi-sdk}
    \item wasi-libc: \url{https://github.com/martin-fink/wasi-libc}
    \item PolyBench/C: \url{https://github.com/martin-fink/polybench-c}
\end{itemize}

\section{Building}
\label{sec:building}

All the projects contain a \texttt{flake.nix} that can be used to start a development shell containing all the dependencies.
On systems without Nix, these need to be installed manually.

\subsection{LLVM Toolchain}
\label{subsec:llvm-toolchain}

The following commands can be used to bootstrap the \ac{WASM} compiler toolchain, including the libc.

\begin{lstlisting}[label={lst:building-sdk}]
    git clone --recurse-submodules https://github.com/martin-fink/wasi-sdk.git
    cd wasi-sdk
    make package
    cd dist
    tar -xzf wasi-sdk-20.38g2b3e8f68d320-linux.tar.gz
    tar -xzf libclang_rt.builtins-wasm64-wasi-20.38g2b3e8f68d320.tar.gz
    mv lib wasi-sdk-20.38g2b3e8f68d320/lib/clang/18/
\end{lstlisting}

\subsection{Wasmtime}
\label{subsec:building-wasmtime}

We are building wasmtime for Android, as the Pixel 8 is the only device supporting \ac{MTE} at the time of writing.
The same steps are also possible if you are building for Linux.
In any case, you need a C compiler and linker for the corresponding target.

\begin{lstlisting}[label={lst:building-wasmtime}]
    git clone --recurse-submodules https://github.com/TUM-DSE/wasmtime-mte.git wasmtime
    cd wasmtime
\end{lstlisting}

\subsubsection{Building for Android}

You need to install the Android NDK\footnote{\url{https://developer.android.com/ndk}}.
Then, run the following commands:

\begin{lstlisting}[label={lst:configuring-wasmtime-android}]
    rustup target add aarch64-linux-android
    mkdir .cargo
    echo << EOF
    [env]
    [target.aarch64-linux-android]
    linker = "/path/to/android/aarch64-linux-android34-clang"
    ar = "/path/to/android/llvm-ar"
    rustflags = ["-C", "target-feature=+mte"]
    EOF >> .cargo/config.toml
    cargo build --release --target aarch64-linux-android
\end{lstlisting}

\subsubsection{Building for Linux}

You need a cross-compiler for aarch64 Linux.
Then, run the following commands:

\begin{lstlisting}[label={lst:configuring-wasmtime-linux}]
    rustup target add aarch64-unknown-linux-gnu
    mkdir .cargo
    echo << EOF
    [env]
    CC_aarch64-unknown-linux-gnu = "aarch64-linux-gnu-gcc"
    CC_aarch64-unknown-linux-musl = "aarch64-linux-gnu-gcc"

    [target.aarch64-unknown-linux-gnu]
    linker = "aarch64-linux-gnu-gcc"
    rustflags = ["-C", "target-feature=+mte"]

    [target.aarch64-unknown-linux-musl]
    linker = "aarch64-linux-gnu-gcc"
    rustflags = ["-C", "target-feature=+mte"]
    EOF >> .cargo/config.toml
    cargo build --release --target aarch64-unknown-linux-gnu
\end{lstlisting}

\section{Running Programs}
\label{sec:running-programs}

\subsection{Compiling with Memory Safety}\label{subsec:compiling-with-memory-safety}

\begin{lstlisting}[label={lst:compiling-to-wasm}]
    WASI_SDK=wasi-sdk-20.38g2b3e8f68d320
    "./$WASI_SDK/bin/clang" \
      -mmem-safety \
      -Os \
      -fsanitize=wasm-memsafety \
      --sysroot "$WASI_SDK/share/wasi-sysroot" \
      main.c \
      -o main.wasm
\end{lstlisting}

\subsection{Running with Wasmtime}\label{subsec:running-with-wasmtime}

\begin{lstlisting}[label={lst:running-wasm}]
    ./wasmtime run \
      -W memory64 \
      -W mem-safety \
      -C mte-bounds-checks \
      -C mte \
      main.wasm
\end{lstlisting}
