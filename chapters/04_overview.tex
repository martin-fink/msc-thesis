\chapter{Overview}
\label{ch:overview}

In this thesis, we present a design for an extension that enhances memory safety in \ac{WASM} built on top of the \ac{WASM} 64-bit memory proposal\footnote{\url{https://github.com/WebAssembly/memory64}}.
The extension is created to be minimally invasive and implementable using various techniques, including hardware extensions such as \ac{MTE} or \ac{PAC}, capability-based architectures like \ac{CHERI}~\cite{woodruff2014cheri}, or software-based solutions similar to \ac{ASAN}~\cite{serebryany2012addresssanitizer} or \ac{HWASAN}~\cite{serebryany2018memory} (see \cref{ch:design}).

We implemented a prototype of this design, which includes a sanitizer and code generation backend based on LLVM~\cite{lattner2004llvm}, a modified standard library based on \ac{WASI}, and an implementation based on \ac{MTE} in wasmtime\footnote{\url{https://wasmtime.dev/}} (see \cref{ch:implementation}).
Additionally, we explored and implemented a technique to sandbox \ac{WASM} programs using \ac{MTE}.
This technique reduces the overhead of 64-bit \ac{WASM} programs significantly and can be combined with the \ac{MTE}-based memory safety implementation (see \cref{ch:design,ch:implementation}).

We analyzed and benchmarked various aspects of our implementation, including 32-bit and 64-bit \ac{WASM}, and the \ac{MTE} implementation on real hardware (see \cref{ch:eval}).
