\chapter{Related Work}
\label{ch:related}

This chapter reviews existing efforts in ensuring memory safety, focusing on \ac{WASM} and C/C++ compiled directly to machine code.
We explore notable projects in these areas, comparing their approaches with our research to highlight our distinct contributions.

\section{Memory Safety for WebAssembly}
\label{sec:related-memory-safety-for-webassembly}

This thesis builds upon existing efforts in the field of memory safety for \ac{WASM}.
Here, we examine notable projects aiming to achieve similar goals and highlight the distinct contributions of our research.

\subsection{MS-WASM}
\label{subsec:ms-wasm}

A significant project in this domain is MS-WASM, a memory safety extension for WASM introduced by \citeauthor*{disselkoen2019position} and further developed by \citeauthor*{michael2023mswasm}~\cite{disselkoen2019position,michael2023mswasm}.
MS-WASM introduces a new \textit{segment memory} distinct from the linear memory, preventing accesses through arbitrary offsets.
The segment memory relies on access through unforgeable \textit{handles}, akin to CHERI pointers~\cite{woodruff2014cheri}.

\paragraph{Key Differences}
This thesis adopts a different approach by enabling a gradual migration of memory segments into the linear memory.
This preserves compatibility with unmodified code, as only the allocation of memory regions has to be changed.
Memory accesses are still performed through integers, not pointers.
We do not implement intra-object memory safety to allow implementation with \ac{MTE}.
While MS-WASM proposes a technique to implement intra-object safety with a tagged-memory approach (shading), this is not supported by \ac{MTE}.
We thus choose this tradeoff, as utilizing \ac{MTE} allows our implementation to run with significantly lower overhead on devices supporting this technology.

\subsection{RichWasm}
\label{subsec:richwasm}

RichWasm is a richly-typed intermediate language for safe memory interactions between languages with varying memory management models~\cite{paraskevopoulou2024richwasm}.
It allows for static detection of memory safety violations and is especially beneficial for mixed-language interoperability.

RichWasm is intended for use as a compilation target for safe languages like Rust or OCaml, which have strong memory safety guarantees encoded in their type systems.
Languages like C, which lack information for static type safety analysis, are not directly supported by RichWasm's type-driven safety model.

\subsection{Pointer Authentication}
\label{subsec:related-pointer-authentication}

\citeauthor*{rehde2023wasm} has worked on implementing pointer authentication primitives~\cite{rehde2023wasm}.
In their work, they add pointer authentication primitives for data pointers backed by ARM's \ac{PAC} to the memory safety extension described in this thesis.
This complements our work by providing additional protection mechanisms against pointer corruption.

\section{Memory Safety for C}
\label{sec:related-memory-safety}

Various projects exist that work on providing memory safety for C.
In this section discuss several approaches to provide memory safety for C outside the context of \ac{WASM}, such as language extensions (\cref{subsec:extensions-to-the-c-language}), instrumentation-based methods (\cref{subsec:automatic-approaches}), and hardened libraries (\cref{subsec:hardened-memory-allocators}).

\subsection{Memory-Safe C Language Dialects}
\label{subsec:extensions-to-the-c-language}

Several projects such as CCured~\cite{necula2002ccured} or Cyclone~\cite{jim2002cyclone} implement memory safety by extending the C language.
Their approach adds information about allocations, but requires manual changes to existing code.
CHERI C/C++~\cite{watson_cheri_2020} goes into a similar direction, widening pointers to 128\,bits.
CHERI C/C++ can be compiled in one of two modes: (1) purecap mode, where all pointers are capabilities and the compiler restricts them to allocations, automatically providing memory safety and (2) hybrid mode, where pointers remain 64\,bits wide.
Programmers can annotate pointers, which transforms them into capabilities, allowing capability-aware and unmodified code to be mixed.
All of these approaches break binary compatibility by storing metadata in a fat pointer representation.

\subsection{Instrumentation-Based Memory Safety}
\label{subsec:automatic-approaches}

The following projects automate memory safety without source code changes, employing various strategies to detect and prevent memory errors~\cite{serebryany2012addresssanitizer,serebryany2023gwp,nethercote2007valgrind}.
We can divide these approaches into three major categories: (1) trip-wire-based, (2) object-based, and (3) pointer-based.

\subsubsection{Trip-Wire-Based Approach}

Trip-Wire-Based projects guard zones around allocations to catch memory safety errors.
Some projects that fall into this category are \ac{ASAN}~\cite{serebryany2012addresssanitizer}, GWP-ASan~\cite{serebryany2023gwp}, or Valgrind Memcheck~\cite{nethercote2007valgrind}.
In the case of \ac{ASAN}, $\sfrac{1}{8}$ of the whole address space is mapped as shadow memory, with one bit of shadow memory representing the state of one byte of memory.
When allocating memory, the allocation is padded and guard zones around the allocation are marked as inaccessible.
When freeing memory, the whole memory block is marked as inaccessible.
The sanitizer inserts checks before each memory access that checks the address currently being accessed.
\Ac{ASAN} was built as a debugging tool to be used when running tests or fuzzing programs, as it has an average overhead of 73\,\%.

GWP-ASan~\cite{serebryany2023gwp} is intended to be deployed in production.
It adopts a probabilistic approach to memory safety by sampling a subset of allocations and placing them in a guarded memory region to detect spatial and temporal memory safety violations for those allocations.
Its low selection probability minimizes runtime overhead.
The goal is to uncover hard-to-reproduce memory bugs triggered by real-world user behavior not covered by fuzzing or tests.

In contrast, Valgrind Memcheck utilizes dynamic binary instrumentation, which does not require recompilation from source.

\subsubsection{Object-Based Approach}

For object-based approaches, the metadata is associated with the allocated object.
A notable project in this area is Baggy Bounds Checking~\cite{akritidis2009baggy}, which aligns and pads allocations to powers of two to speed up bounds checks at runtime.
Metadata about the allocation can be efficiently retrieved thanks to the alignment of the allocation.
Usually, pointers are checked when doing pointer arithmetic, not when dereferencing pointers.
This poses a challenge for C, as the language permits out-of-bounds pointers, as long as they are not dereferenced.
Intra-object safety is more challenging for the object-based-approaches, as enforcing those additional constraints may require more sophisticated data structures.
Baggy Bounds Checking does not support this intra-object safety.

\subsubsection{Pointer-Based Approach}

In contrast, pointer-based approaches keep track of the pointer bounds.
These can be either stored in the pointer, as fat pointers~\cite{watson_cheri_2020} or in unused bits~\cite{serebryany2018memory} or in an external data structure~\cite{nagarakatte2009softbound}.
One project in this domain is SoftBound~\cite{nagarakatte2009softbound}.
It keeps track of the upper and lower bounds of each pointer, instrumenting memory accesses to check if the pointer is within its bounds.
This allows to implement intra-object memory safety.
Additionally, creating out-of-bounds pointers is not an issue, as pointer arithmetic is not checked.
SoftBound keeps track of pointers stored in memory in the form of a separate data structure, where bounds are looked up and stored to when loading and storing pointers from and to memory.
Similarly, when passing pointers to and from functions, the bounds need to be propagated as additional arguments and return values.

CHERI~\cite{woodruff2014cheri} is an implementation of a pointer-based approach with hardware support.
It extends pointers to 128\,bits, including permission and compressed bounds.
Pointer dereferences are checked by hardware, promising much better performance than software-based checks.
Currently, hardware exists in the form of the ARM Morello board~\cite{UCAM-CL-TR-982}, a limited-production development board, shipped to selected academic and industry partners.

\subsubsection{Hybrid Approaches}

We consider memory-tagging-based approaches as a hybrid of the two approaches before.
\Ac{HWASAN}~\cite{serebryany2018memory} or \ac{MTE} both associate metadata with the pointer in its upper, unused bits, and with objects by assigning tags to memory.
In the case of \ac{HWASAN}, only memory accesses are instrumented, while memory accesses are checked by hardware with \ac{MTE}.
This again allows out-of-bounds pointers to exist if they are not dereferenced.
However, intra-object safety is not possible to implement, similar to object-based approaches.
In the case of both \ac{HWASAN} and \ac{MTE}, binary compatibility is preserved, as the hardware ignores extra metadata in the upper bits of pointers, thus allowing uninstrumented code to handle pointers with and without metadata.

\subsection{Hardened Memory Allocators}
\label{subsec:hardened-memory-allocators}

A few memory allocators have implemented support for \ac{MTE}.
These provide probabilistic memory safety against both spatial and temporal memory safety, as long as no tag collisions occur.

\begin{itemize}
    \item Scudo Hardened Allocator (used in Android): A security-oriented allocator providing defense mechanisms against heap-based vulnerabilities~\cite{scudo_allocator}.
    \item Chrome's PartitionAlloc: A partitioning allocator focusing on security and efficiency for multithreaded environments~\cite{chrome_partition_alloc}.
    \item glibc's Ptmalloc: The GNU C library's standard memory allocator, with evolving experimental support for \ac{MTE}~\cite{glibc_ptmalloc}.
\end{itemize}

\paragraph{}
The Cling memory allocator~\cite{akritidis2010cling} uses a different approach to prevent use-after-free exploits by placing heap metadata out-of-band and reusing memory only for objects of the same type.
It achieves this by analyzing the call stack to determine the type of data being allocated and works as a drop-in replacement, without any code changes.
