\chapter{Related Work}
\label{ch:related}

\section{Memory Safety for WebAssembly}
\label{sec:related-memory-safety-for-webassembly}

This thesis builds upon existing efforts in the field of memory safety for \ac{WASM}.
Here, we examine notable projects aiming to achieve similar goals and highlight the distinct contributions of our research.

\subsection{MS-WASM}
\label{subsec:ms-wasm}

A significant project in this domain is MS-WASM, a memory safety extension for WASM introduced by \citeauthor*{disselkoen2019position} and further developed by \citeauthor*{michael2023mswasm}~\cite{disselkoen2019position,michael2023mswasm}.
MS-WASM introduces a new \textit{segment memory} distinct from the linear memory, preventing accesses through arbitrary offsets.
The segment memory relies on access through unforgeable \textit{handles}, akin to CHERI pointers~\cite{woodruff2014cheri}.

\paragraph{Key Differences}
This thesis adopts a different approach by enabling a gradual migration of memory segments into the linear memory.
This preserves compatibility with unmodified code, as only the allocation of allocation has to be changed.
Memory accesses are still performed through integers, not pointers.
We do not implement intra-memory safety to allow implementation with \ac{MTE}, which uses colors for memory access control rather than the shading required for intra-object safety.
Utilizing \ac{MTE} allows our implementation to run with significantly lower overhead on devices supporting this technology.

\subsection{Pointer Authentication}
\label{subsec:related-pointer-authentication}

\citeauthor*{rehde2023wasm} has worked on implementing pointer authentication primitives~\cite{rehde2023wasm}.
In their work, they add pointer authentication primitives backed by ARM's \ac{PAC} to the memory safety extension described in this thesis.
This complements our work by providing additional protection mechanisms against pointer corruption.

