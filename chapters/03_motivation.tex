\chapter{Motivation}
\label{ch:motivation}

While WebAssembly provides strong safety and security guarantees, as discussed in \cref{sec:wasm}, these mainly guarantee safety for the host, not the program itself.
In~\cite{lehmann2020everything}, \citeauthor*{lehmann2020everything} show that while some attack surfaces, such as injecting shellcode or jumping to arbitrary addresses, are mitigated by the design of WebAssembly, others, such as buffer overflows or write accesses to static, read-only data is possible and being used to exploit programs running in the wild.
These need to be mitigated at the language level by rewriting software in a safe language such as Rust, manually inserting bounds checks, which is error-prone, or inserting checks using the compiler and sanitizing the code.

Additionally, bugs like {CVE-2023-4863}~\cite{CVE-2023-4863} continue to be exploited, showing that memory safety is not a solved problem.
While they do not escape the WebAssembly sandbox, they pose a security risk to the programs themselves.
In C, the use of unsafe primitives or bugs, such as missing bounds checks, can be exploited by the attackers, e.g., by overwriting a variable to elevate their privileges.
In \cref{fig:vulnerable-overflow}, the lack of bounds checks allows an attacker controlling the variable \texttt{input} to write beyond the allocation of \texttt{buf} and overwrite \texttt{str}.

\begin{figure}[h]
    \centering
    \begin{lstlisting}[frame=h,style=customc,label={lst:vulnerable-overflow}]
void foo(char *input) {
    char buf[32];
    const char str = "Hello, World!";
    strcpy(buf, input);
}
    \end{lstlisting}
    \caption{Vulnerable overflow}
    \label{fig:vulnerable-overflow}
\end{figure}

To contain potentially malicious programs within their sandbox, several different techniques may be used, as discussed in \cref{subsec:webassembly-sandbox}.
Most implementations rely on virtual memory and guard pages to contain memory access when possible.
However, in cases where virtual memory cannot be used, e.g., when running 64\,bit WebAssembly programs, the engine needs to fall back to software-based bounds checks, bringing a significant performance penalty.
In our measurements, switching to 64\,bit \ac{WASM} resulted in a roughly 6-8\,\% overhead on out-of-order CPUs, which can speculate bounds checks, and \todo{x}\,\% overhead on in-order CPUs (see detailed evaluation in \cref{sec:performance-overheads}).
The fallback to software-based bounds checks is thus especially painful when running on low-power in-order cores when running 64\,bit \ac{WASM} or in environments without an operating system, such as embedded devices.

